\documentclass[letterpaper]{twentyonesecondcv} % a4paper for A4
\usepackage[english,brazil]{babel}
\usepackage{url}
\profilepic{perfil.jpg} % Profile picture
\cvjobtitle{Trabalho e Carreira} % Job title/career
\cvname{Victor Carreira} % Your name

%% Mandatory informations used by \makeinfoprofile. To hide these fields leave the contents of the macro empty (e.g. '\cvsitepersonal{}' instead than '\cvsitepersonal{en.wikipedia.org}')
%% See https://tex.stackexchange.com/a/692525/109031
\cvbirthdate{13 de Dezembro de 1984}
\cvnumberphone{+55 (21)998288484}
\cvaddressurl{https://maps.app.goo.gl/SyUn58bekAuLpxys5} 
\cvaddress{Av. Gal. Milton Tavares de Souza, s/n - São Domingos, Niterói - RJ, 24210-310} % Short address/location, use \newline if more than 1 line is required
\cvsitepersonal{carreiras.netlify.com}       % personal site
\cvstackoverflow{Victor Carreira - 11963659}
\cvlinkedin{Victor Carreira}
\cvskypeurlbase{join.skype.com/invite/id-random-skype-invite} % Skype 
\cvskypeurl{Victor Ribeiro Carreira}
\cvgithub{Victor Carreira}     
\cvmail{victorcarreira@id.uff.br}

\textfootersidenote{A side footer text that appears on two pages...}{on two rows.}

\pagenumber{...Page}{of 2}

\begin{document}
\sidesection{
    \makeheaderprofile
    % \makeheaderprofilenoimg
    \makeinfoprofile
    \aboutme{Pesquisador que trabalha com computação científica, modelagem e ciência de dados. Geologia (2012). Mestrado (2015) métodos potenciais e magnetotelúrico. Doutorado (2021) IA / metaheurística.}
   % \customsidesection{Header profile section}{It's possible to hide the profile picture using \textbackslash\texttt{makeheaderprofilenoimg} instead than \textbackslash\texttt{makeheaderprofile} (you could remove  \textbackslash\texttt{profilepic\{image.png\}} then).}
   % \customsidesection{Info profile section}{The command \textbackslash\texttt{makeinfoprofile} doesn't use empty macros \textbackslash\texttt{cvsite*\{\}} (e.g. \textbackslash\texttt{cvsitepersonal\{\}} instead than \textbackslash\texttt{cvsitepersonal\{example.com\}}).}
   % \customsidesection{About the info profile commands}{If not empty the \textbackslash\texttt{cvaddressurl\{\}} command create a custom hyperlink containing the \textbackslash\texttt{cvaddress\{Address, Nation\}} text. Same for \textbackslash\texttt{cvskypeurlbase\{join.skype.com/...\}} and \textbackslash\texttt{cvskypeurl\{skype-username\}} (visit \colorhrefcustom{https://support.skype.com/en/faq/FA34802/}{skype FAQs}) to create an invite url.}
    
    \makefootersidenodevfill
    
}
% %%%%%%%%%%%%%%%%%%%%%%%%%%%%%%%%%%%%%%%%%%%%%%%%%%%%%%%%%%%%%
% for some reason it's impossible to have a new line here...
\mainsection{
    
    \section{Resumo Bibiográfico}
		Atualmente trabalha como pesquisador de pós-doutorado na Universidade Federal
		Fluminense integrando a equipe do projeto Ressurgência. Possui doutorado em Geofísica
		pelo Observatório Nacional, na área de geofísica aplicada à Geofísica de Exploração com
		ênfase em meta-heurística e métodos de inteligência artificial aplicados à dados de
		perfilagem geofísica, em Bacias Sedimentares (2021).Mestrado em Geofísica pelo
		Observatório Nacional na área de Métodos Potencias e Eletromagnéticos (2015) com experiência em
		levantamentos de campo na área de geofísicos e graduação em Geologia pela Universidade Federal do Rio de Janeiro (2012). Em sua
		graduação, trabalhou como estagiário (bolsista COPPETEC), no projeto "Aplicação da
		Bioestratigrafia da Radiolários ao Refinamento Estratigráfico do Cretáceo e Paleógeno nas
		Bacias Brasileiras", financiado pelo BPA/Cenpes/Petrobras/ e vinculado ao LAFO - UFRJ.
		Em seu mestrado, desenvolveu estudos relativos ao embasamento da Bacia do Paraná e a
		Bacia Sedimentar correlata através do uso de Métodos Potenciais e Magnetotelúrico. Em
		seu doutorado, desenvolveu uma nova metodologia aplicadas à área de Inteligência
		Artificial para reconhecimento de padrões litológicos e um novo modelo gerador de
		pseudo-poços baseados em seções geológicas e sísmica 2D.
    
    \section{Histórico Educacional}

\begin{twenty}
    \twentyitem
        {2017 - 2021}
        {Doutorado em Geofísica}
        {Observatório Nacional, ON, Brasil}
        {Título: Inteligência Artificial aplicada ao reconhecimento de padrões litológicos. Orientador: Cosme Ferreira da Ponte Neto. Coorientador: Rodrigo Bijani Santos. Bolsista da Fundação Carlos Chagas Filho de Amparo à Pesquisa do Estado do RJ, FAPERJ, Brasil.}
    \twentyitem
        {2013 - 2015}
        {Mestrado em Geofísica}
        {Observatório Nacional, ON, Brasil}
        {Título: Contribuição para o entendimento das estruturas geológicas da Bacia do Paraná utilizando métodos potenciais. Orientador: Emanuele Francesco La Terra. Coorientador: Sergio Luiz Fontes. Bolsista do Conselho Nacional de Desenvolvimento Científico e Tecnológico, CNPq, Brasil. Palavras-chave: Métodos Potenciais. Grande área: Ciências Exatas e da Terra. Setores de atividade: Atividades de Apoio à Extração de Minerais.}
    \twentyitem
        {2007 - 2012}
        {Graduação em Geologia}
        {Universidade Federal do Rio de Janeiro, UFRJ, Brasil}
        {Título: O GÁS DE FOLHELHO: UMA NOVA FRONTEIRA. Orientador: José Mário Coelho. Bolsista da Fundação coordenação de projetos, pesquisa e estudos tecnológicos, COPPETEC, Brasil.}
    \twentyitem
        {2002 - 2004}
        {Curso técnico/profissionalizante}
        {Centro Federal de Educação Tecnológica de Química, CEFETEQ, Brasil}
        {} % No description provided in the .md file 
\end{twenty}

    \section{Certificações}
    
    \begin{twentymid} % Environment for a list with descriptions 
	\twentymiditem{2017}{Métodos numéricos e programação científica com o uso do FORTRAN 95 (120h)}{UFRGS, RS, Brasil}    
        \twentymiditem{2013}{Mapeamento e processamento com o geosoftware (30h)}{Observatório Nacional, ON, Brasil}  
        \twentymiditem{2012}{Ground Penetrating Radar (GPR). Teoria e prática com levantamento de campo (30h)}{USP, SP, Brasil} 
        %\twentymiditem{since 2020}{I learned some interesting things on this \colorhrefcustom{https://wikipedia.it}{website}, then I moved on.}{Org}
        %\twentymiditem{10/2015}{I added some fancy pages on \colorhref{wikipedia.org}}{Org 2}
        %\twentymiditem{1990 - 2015}{Second list element. Nunc rhoncus a nunc ac varius. Integer condimentum sit amet dui sed fringilla.}{Organization name}
	%\twentymiditem{12/1998}{Lorem ipsum dolor sit amet, consectetur adipiscing elit. Nulla suscipit a nisi nec fermentum. Praesent sed leo fringilla. Lorem ipsum dolor sit amet, consectetur adipiscing elit. Nulla suscipit a nisi nec fermentum. Praesent sed leo fringilla}{\colorhrefcustom{https://commons.wikimedia.org/wiki/Main_Page}{wikipedia}}
    \end{twentymid}
    
    \section{Premiações, menções honrosas}
    
    \begin{twentyshort}
      \twentyitemshort
        {2011}
        {Fundação da Empresa Júnior de Geologia da UFRJ Xisto Jr.}
      \twentyitemshort
        {2015}
	    {O papel do geólogo no século XXI - mesa redonda, UFRJ. (menção honrosa)}
      \twentyitemshort
	    {2023}
	    {Prêmio PETROBRAS de inovação - SOFTWARE achillesBR}

    \end{twentyshort}

%% end main section
}
% \newpage

\clearpage  % mandatory to make it work the command '\pagenumber'

% \noindent
\sidesection{
    \makeheaderprofilenoimg
    \customsidesection{Side footer}{Now there are three commands to populate the side footer:

    \begin{itemize}
      \item \textbackslash\texttt{makefooterprofile}, the old way to create a custom side footer (it takes two mandatory arguments)
      \item \textbackslash\texttt{makefootersidenode}, useful for creating the same footer on all existing pages
      \item \textbackslash\texttt{makefootersidenodevfill}
    \end{itemize}

    Use the commands \textbackslash\texttt{pagenumber}\{\}\{\} and \textbackslash\texttt{textfootersidenote}\{\}\{\} to add the page number (e.g. "Page 2 of 2") and the side footer text.
    }

    \customsidesection{Progress bar skill sections}{The command \textbackslash\texttt{customskills\{Section title\}\{\{\},\{\},\{\}\}\{\}} take three argument (the last one is optional). The first argument is the section title, e.g. \texttt{Languages}. The second macro argument is an array of couples made by words and float numbers separated by slashes, e.g. \texttt{\{\{Latin/4.3\},\{Greek/6\}\}}; the number of course represents how much the skill ability is high. This array separator is a comma. Note that the array order is reversed: in the output list created by the example above has the word "Greek" before "Latin".}
    
    %%%%%%%%%%%%%%%%%%%%%%%%%%%%%%%%%%%%%%%%%%%%%%%%%%%%%%%%%%%%%%
    %%%%%%Skill bar section, each skill must have a value between 0 an 6 (float)%%%%%%%
    %%%%%%%%%%%%%%%%%%%%%%%%%%%%%%%%%%%%%%%%%%%%%%%%%%%%%%%%%%%%%%
    \customskills{Languages skill}{{Latin/4},{Greek/6}}{}

    \customskills{Card games}{{Black Jack/5},{Freecell/2},{Spider/3}}{Scale: 0 (basic skills) - 6 (expert).}
    
    \makefooterprofile{Alternative footer note (only on page two), row one.}{Alt. footer note (row two)...}
    
    \makefootersidenode
    
}
% %%%%%%%%%%%%%%%%%%%%%%%%%%%%%%%%%%%%%%%%%%%%%%%%%%%%%%%%%%%%%
% for some reason there is no way to have a new line here...
\mainsection{
    
    \section{Experiences}
    
    \begin{twenty}
      \twentyitem
        {since 2019}
        {My last work experience}
        {Work company}
        {Lorem ipsum dolor sit amet, consectetur adipiscing elit. Nulla suscipit a nisi nec fermentum. Praesent sed leo fringilla, eleifend diam vel, ultricies mauris. Nunc interdum diam vel magna egestas posuere. Mauris sodales urna vitae neque imperdiet tempus. Nunc rhoncus a nunc ac varius. Integer condimentum sit amet dui sed fringilla. Lorem ipsum dolor sit amet, consectetur adipiscing elit. Nulla suscipit a nisi nec fermentum. Praesent sed leo fringilla, eleifend diam vel, ultricies mauris. Nunc interdum diam vel magna egestas posuere. Mauris sodales urna vitae neque imperdiet tempus. Nunc rhoncus a nunc ac varius. Integer condimentum sit amet dui sed fringilla.}
      \twentyitem
        {1993}
        {Other side work.}
        {ACME Corp.}
        {Lorem ipsum dolor sit amet, consectetur adipiscing elit. Nulla suscipit a nisi nec fermentum. Praesent sed leo fringilla.}
    \twentyitem
        {07/1993}
        {Side work.}
        {ACME Corp.}
        {Nunc interdum diam vel magna egestas posuere. Mauris sodales urna vitae neque imperdiet tempus. Nunc rhoncus a nunc ac varius. Integer condimentum sit amet dui sed fringilla. Lorem ipsum dolor sit amet, consectetur adipiscing elit. Nulla suscipit a nisi nec fermentum. Praesent sed leo fringilla, eleifend diam vel, ultricies mauris. Nunc interdum diam vel magna egestas posuere. Mauris sodales urna vitae neque imperdiet tempus. Nunc rhoncus a nunc ac varius. Integer condimentum sit amet dui sed fringilla.}
    \twentyitem
        {1981}
        {My first work experience.}
        {Convenience Store}
        {Lorem ipsum dolor sit amet, consectetur adipiscing elit. Nulla suscipit a nisi nec fermentum. Praesent sed leo fringilla, eleifend diam vel, ultricies mauris.}
    \end{twenty}
    
    \section{Other informations}
    
    Other information section. In sed ligula a turpis tristique interdum a ut risus. Nullam at euismod leo, ac molestie urna. Curabitur enim felis, ultricies quis dui ac, bibendum auctor dui. Proin aliquet nulla arcu, at pulvinar eros eleifend at. Aliquam erat volutpat. Duis augue mauris, aliquam sit amet lacinia ut, egestas sit amet est. Aliquam erat lacus, scelerisque non laoreet eu, commodo eget turpis. Nunc nunc mauris, lacinia sed semper quis, venenatis eget enim. Proin viverra in nibh at tincidunt.
    
    Lorem ipsum dolor sit amet, consectetur adipiscing elit. Donec euismod id nisi nec dapibus. Nunc sed cursus nisi, in feugiat elit. Orci varius natoque penatibus et magnis dis parturient montes, nascetur ridiculus mus. Praesent dictum vehicula est non dapibus. Praesent id tincidunt quam. Morbi pharetra, lorem non faucibus vestibulum, turpis eros blandit felis, at mollis diam metus in elit. In venenatis pharetra leo vel elementum. Lorem ipsum dolor sit amet, consectetur adipiscing elit. Ut non nulla pellentesque, vestibulum lorem in, vehicula velit. Integer vitae fringilla ipsum. Donec vel pretium libero.
    
    In sed ligula a turpis tristique interdum a ut risus. Nullam at euismod leo, ac molestie urna. Curabitur enim felis, ultricies quis dui ac, bibendum auctor dui. Proin aliquet nulla arcu, at pulvinar eros eleifend at. Aliquam erat volutpat. Duis augue mauris, aliquam sit amet lacinia ut, egestas sit amet est. Aliquam erat lacus, scelerisque non laoreet eu, commodo eget turpis. Nunc nunc mauris, lacinia sed semper quis, venenatis eget enim. Proin viverra in nibh at tincidunt.
    
    Integer molestie, lectus in vehicula consequat, mauris nisi dignissim lacus, et vestibulum ante nisl eu ante. In laoreet eros sit amet justo tempor maximus. Cras quam velit, feugiat vel risus vitae, tincidunt elementum.

}
\end{document}
