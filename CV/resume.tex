\documentclass[letterpaper]{twentyonesecondcv} % a4paper for A4
\usepackage[english,brazil]{babel}
\usepackage{url}
\profilepic{perfil.jpg} % Profile picture
\cvjobtitle{Trabalho e Carreira} % Job title/career
\cvname{Victor Carreira} % Your name

%% Mandatory informations used by \makeinfoprofile. To hide these fields leave the contents of the macro empty (e.g. '\cvsitepersonal{}' instead than '\cvsitepersonal{en.wikipedia.org}')
%% See https://tex.stackexchange.com/a/692525/109031
\cvbirthdate{13 de Dezembro de 1984}
\cvnumberphone{+55 (21)998288484}
\cvaddressurl{https://maps.app.goo.gl/SyUn58bekAuLpxys5} 
\cvaddress{Av. Gal. Milton Tavares de Souza, s/n - São Domingos, Niterói - RJ, 24210-310} % Short address/location, use \newline if more than 1 line is required
\cvsitepersonal{carreiras.netlify.com}       % personal site
\cvstackoverflow{Victor Carreira - 11963659}
\cvlinkedin{Victor Carreira}
\cvskypeurlbase{join.skype.com/invite/id-random-skype-invite} % Skype 
\cvskypeurl{Victor Ribeiro Carreira}
\cvgithub{Victor Carreira}     
\cvmail{victorcarreira@id.uff.br}

\textfootersidenote{CVRRICVLVM VITAE}{Victor Carreira} % The text in the footer of the first page

\pagenumber{...Página}{of 2}

\begin{document}
\sidesection{
    \makeheaderprofile
    % \makeheaderprofilenoimg
    \makeinfoprofile
    \aboutme{Pesquisador que trabalha com computação científica, modelagem e ciência de dados. Geologia (2012). Mestrado (2015) métodos potenciais e magnetotelúrico. Doutorado (2021) IA / metaheurística.}
   % \customsidesection{Header profile section}{It's possible to hide the profile picture using \textbackslash\texttt{makeheaderprofilenoimg} instead than \textbackslash\texttt{makeheaderprofile} (you could remove  \textbackslash\texttt{profilepic\{image.png\}} then).}
   % \customsidesection{Info profile section}{The command \textbackslash\texttt{makeinfoprofile} doesn't use empty macros \textbackslash\texttt{cvsite*\{\}} (e.g. \textbackslash\texttt{cvsitepersonal\{\}} instead than \textbackslash\texttt{cvsitepersonal\{example.com\}}).}
   % \customsidesection{About the info profile commands}{If not empty the \textbackslash\texttt{cvaddressurl\{\}} command create a custom hyperlink containing the \textbackslash\texttt{cvaddress\{Address, Nation\}} text. Same for \textbackslash\texttt{cvskypeurlbase\{join.skype.com/...\}} and \textbackslash\texttt{cvskypeurl\{skype-username\}} (visit \colorhrefcustom{https://support.skype.com/en/faq/FA34802/}{skype FAQs}) to create an invite url.}
    
    \makefootersidenodevfill
    
}
% %%%%%%%%%%%%%%%%%%%%%%%%%%%%%%%%%%%%%%%%%%%%%%%%%%%%%%%%%%%%%
% for some reason it's impossible to have a new line here...
\mainsection{
    
    \section{Resumo Bibiográfico}
		Atualmente trabalha como pesquisador de pós-doutorado na Universidade Federal
		Fluminense integrando a equipe do projeto Ressurgência. Possui doutorado em Geofísica
		pelo Observatório Nacional, na área de geofísica aplicada à Geofísica de Exploração com
		ênfase em meta-heurística e métodos de inteligência artificial aplicados à dados de
		perfilagem geofísica, em Bacias Sedimentares (2021).Mestrado em Geofísica pelo
		Observatório Nacional na área de Métodos Potencias e Eletromagnéticos (2015) com experiência em
		levantamentos de campo na área de geofísicos e graduação em Geologia pela Universidade Federal do Rio de Janeiro (2012). Em sua
		graduação, trabalhou como estagiário (bolsista COPPETEC), no projeto "Aplicação da
		Bioestratigrafia da Radiolários ao Refinamento Estratigráfico do Cretáceo e Paleógeno nas
		Bacias Brasileiras", financiado pelo BPA/Cenpes/Petrobras/ e vinculado ao LAFO - UFRJ.
		Em seu mestrado, desenvolveu estudos relativos ao embasamento da Bacia do Paraná e a
		Bacia Sedimentar correlata através do uso de Métodos Potenciais e Magnetotelúrico. Em
		seu doutorado, desenvolveu uma nova metodologia aplicadas à área de Inteligência
		Artificial para reconhecimento de padrões litológicos e um novo modelo gerador de
		pseudo-poços baseados em seções geológicas e sísmica 2D.
    
    \section{Histórico Educacional}

\begin{twenty}
    \twentyitem
        {2017 - 2021}
        {Doutorado em Geofísica}
        {Observatório Nacional, ON, Brasil}
        {Título: Inteligência Artificial aplicada ao reconhecimento de padrões litológicos. Orientador: Cosme Ferreira da Ponte Neto. Coorientador: Rodrigo Bijani Santos. Bolsista da Fundação Carlos Chagas Filho de Amparo à Pesquisa do Estado do RJ, FAPERJ, Brasil.}
    \twentyitem
        {2013 - 2015}
        {Mestrado em Geofísica}
        {Observatório Nacional, ON, Brasil}
        {Título: Contribuição para o entendimento das estruturas geológicas da Bacia do Paraná utilizando métodos potenciais. Orientador: Emanuele Francesco La Terra. Coorientador: Sergio Luiz Fontes. Bolsista do Conselho Nacional de Desenvolvimento Científico e Tecnológico, CNPq, Brasil. Palavras-chave: Métodos Potenciais. Grande área: Ciências Exatas e da Terra. Setores de atividade: Atividades de Apoio à Extração de Minerais.}
    \twentyitem
        {2007 - 2012}
        {Graduação em Geologia}
        {Universidade Federal do Rio de Janeiro, UFRJ, Brasil}
        {Título: O GÁS DE FOLHELHO: UMA NOVA FRONTEIRA. Orientador: José Mário Coelho. Bolsista da Fundação coordenação de projetos, pesquisa e estudos tecnológicos, COPPETEC, Brasil.}
    \twentyitem
        {2002 - 2004}
        {Curso técnico/profissionalizante}
        {Centro Federal de Educação Tecnológica de Química, CEFETEQ, Brasil}
        {} % No description provided in the .md file 
\end{twenty}

    \section{Certificações}
    
    \begin{twentymid} % Environment for a list with descriptions 
	\twentymiditem{2017}{Métodos numéricos e programação científica com o uso do FORTRAN 95 (120h)}{UFRGS, RS, Brasil}    
        \twentymiditem{2013}{Mapeamento e processamento com o geosoftware (30h)}{Observatório Nacional, ON, Brasil}  
        \twentymiditem{2012}{Ground Penetrating Radar (GPR). Teoria e prática com levantamento de campo (30h)}{USP, SP, Brasil} 
        %\twentymiditem{since 2020}{I learned some interesting things on this \colorhrefcustom{https://wikipedia.it}{website}, then I moved on.}{Org}
        %\twentymiditem{10/2015}{I added some fancy pages on \colorhref{wikipedia.org}}{Org 2}
        %\twentymiditem{1990 - 2015}{Second list element. Nunc rhoncus a nunc ac varius. Integer condimentum sit amet dui sed fringilla.}{Organization name}
	%\twentymiditem{12/1998}{Lorem ipsum dolor sit amet, consectetur adipiscing elit. Nulla suscipit a nisi nec fermentum. Praesent sed leo fringilla. Lorem ipsum dolor sit amet, consectetur adipiscing elit. Nulla suscipit a nisi nec fermentum. Praesent sed leo fringilla}{\colorhrefcustom{https://commons.wikimedia.org/wiki/Main_Page}{wikipedia}}
    \end{twentymid}
    
    \section{Premiações, menções honrosas}
    
    \begin{twentyshort}
      \twentyitemshort
        {2011}
        {Fundação da Empresa Júnior de Geologia da UFRJ Xisto Jr.}
      \twentyitemshort
        {2015}
	    {O papel do geólogo no século XXI - mesa redonda, UFRJ. (menção honrosa)}
      \twentyitemshort
	    {2023}
	    {Prêmio PETROBRAS de inovação - SOFTWARE achillesBR}

    \end{twentyshort}

%% end main section
}
% \newpage

\clearpage  % mandatory to make it work the command '\pagenumber'

% \noindent
\sidesection{
    \makeheaderprofilenoimg
    %\makeinfoprofile
    %%%%%%%%%%%%%%%%%%%%%%%%%%%%%%%%%%%%%%%%%%%%%%%%%%%%%%%%%%%%%%
    %%%%%%Skill bar section, each skill must have a value between 0 an 6 (float)%%%%%%%
    %%%%%%%%%%%%%%%%%%%%%%%%%%%%%%%%%%%%%%%%%%%%%%%%%%%%%%%%%%%%%%
    \customskills{Linguagem estrangeira }{{Inglês/9},{Latim/5},{Espanhol/3},{Francês/2},{Polonês/1}}{}

    \customskills{Linguagem de programação}{{FORTRAN/7},{Python/7},{Matlab e Octave/5},{SQL/4}, {Markdown/5}, {Git/7}, {TeX/7}, {ShellScript/4},{Julia/0},{C++/1},{HTML/2}}{Escala: acima da metade da barra de habilidades (Experiente)}

    
    %\makefooterprofile{Alternative footer note (only on page two), row one.}{Alt. footer note (row two)...}
    
    \makefootersidenode
    
}
% %%%%%%%%%%%%%%%%%%%%%%%%%%%%%%%%%%%%%%%%%%%%%%%%%%%%%%%%%%%%%
% for some reason there is no way to have a new line here...
\mainsection{
    
    \section{Experiências de trabalho}
    
    \begin{twenty}
      \twentyitem
        {Desde 2023}
        {Minha atual posição}
        {Universidade Federal Fluminense, UFF, Brasil}
        {Pesquisador de pós-doutorado no projeto Ressurgência fase IV. Desenvolvimento de metodologias e rotinas de programação na determinação de valores de conteúdo orgânico total em paleoambientes lacustres do Cretáceo. }
        \twentyitem
        {2022-2023}
        {Última posição}
        {Universidade Federal Fluminense, UFF, Brasil}
        {Pesquisador de pós-doutorado no projeto Ressurgência fase III. Desenvolvimento de Software para a determinação de valores de conteúdo orgânico total em ambientes e paleoambientes marinhos.}
    \twentyitem
        {2017-2021}
        {Posição Passada}
        {Observatório Nacional, ON, Brasil}
        {Doutorado em Geofísica. Desenvolvimento de uma nova metodologia aplicadas à área de Inteligência Artificial para reconhecimento de padrões litológicos e um novo modelo gerador de pseudo-poços baseados em seções geológicas e sísmica 2D.}
    \twentyitem
        {2013-2015}
        {Posição Passada}
        {Observatório Nacional, ON, Brasil}
        {Mestrado em Geofísica. Contribuição para a discretização de estruturas geológicas do embasamento de bacias sedimentares cratônicas utilizando métodos potenciais e magnetotelúricos. Aquisição de dados de campo e processamento de dados.}    
        \twentyitem
        {2007-2012}
        {Posição Passada}
        {Universidade Federal do Rio de Janeiro, UFRJ, Brasil}
        {Graduação em Geologia. Estágio no projeto "Aplicação da Bioestratigrafia da Radiolários ao Refinamento Estratigráfico do Cretáceo e Paleógeno nas Bacias Brasileiras", financiado pelo BPA/Cenpes/Petrobras/ e vinculado ao LAFO - UFRJ. E fundação da empresa Júnior de Geologia Xisto Jr.}
        \twentyitem
        {2005-2007}
        {Posição Passada}
        {TECMA, Brasil}
        {Empresa de consultoria química e ambiental. Análise química instrumental de água e solo por abosorção atômica e cromatografia gasosa e líquida de amostras de solo e água.}
        \twentyitem
        {2004-2005}
        {Posição Passada}
        {Analytical Solutions, Brasil}
        {Estágio técnico em amostragem e análise química instrumental de água e de solo por abosorção atômica e por cromatografia gasosa e líquida de amostras de solo e água.}
        \twentyitem
        {2002-2003}
        {Posição Passada}
        {Escola Técnica Federal de Química - CEFETEQ, Brasil}
        {Curso técnico em Química Ambiental.}
    \end{twenty}
    
    
    

%\section{Other informations}
    
%    Other information section. In sed ligula a turpis tristique interdum a ut risus. Nullam at euismod leo, ac molestie urna. Curabitur enim felis, ultricies quis dui ac, bibendum auctor dui. Proin aliquet nulla arcu, at pulvinar eros eleifend at. Aliquam erat volutpat. Duis augue mauris, aliquam sit amet lacinia ut, egestas sit amet est. Aliquam erat lacus, scelerisque non laoreet eu, commodo eget turpis. Nunc nunc mauris, lacinia sed semper quis, venenatis eget enim. Proin viverra in nibh at tincidunt.
%    
%    Lorem ipsum dolor sit amet, consectetur adipiscing elit. Donec euismod id nisi nec dapibus. Nunc sed cursus nisi, in feugiat elit. Orci varius natoque penatibus et magnis dis parturient montes, nascetur ridiculus mus. Praesent dictum vehicula est non dapibus. Praesent id tincidunt quam. Morbi pharetra, lorem non faucibus vestibulum, turpis eros blandit felis, at mollis diam metus in elit. In venenatis pharetra leo vel elementum. Lorem ipsum dolor sit amet, consectetur adipiscing elit. Ut non nulla pellentesque, vestibulum lorem in, vehicula velit. Integer vitae fringilla ipsum. Donec vel pretium libero.
%    
%    In sed ligula a turpis tristique interdum a ut risus. Nullam at euismod leo, ac molestie urna. Curabitur enim felis, ultricies quis dui ac, bibendum auctor dui. Proin aliquet nulla arcu, at pulvinar eros eleifend at. Aliquam erat volutpat. Duis augue mauris, aliquam sit amet lacinia ut, egestas sit amet est. Aliquam erat lacus, scelerisque non laoreet eu, commodo eget turpis. Nunc nunc mauris, lacinia sed semper quis, venenatis eget enim. Proin viverra in nibh at tincidunt.
%    
%    Integer molestie, lectus in vehicula consequat, mauris nisi dignissim lacus, et vestibulum ante nisl eu ante. In laoreet eros sit amet justo tempor maximus. Cras quam velit, feugiat vel risus vitae, tincidunt elementum.
%
}
\end{document}
